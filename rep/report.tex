%% LyX 2.0.0rc2 created this file.  For more info, see http://www.lyx.org/.
%% Do not edit unless you really know what you are doing.
\documentclass[english]{article}
\usepackage[T1]{fontenc}
\usepackage[latin9]{inputenc}
\usepackage{setspace}
\onehalfspacing

\makeatletter

%%%%%%%%%%%%%%%%%%%%%%%%%%%%%% LyX specific LaTeX commands.
\newcommand{\noun}[1]{\textsc{#1}}

\makeatother

\usepackage{babel}
\begin{document}

\title{Design and Implementation of a MongoDB Driver for Prolog}


\author{Sebastian Lundstr�m}
\maketitle
\begin{abstract}
Prolog is good stuff, and MongoDB ...

\tableofcontents{}
\end{abstract}

\section{Diary (not part of the report)}
\begin{itemize}
\item March 29. Wrote a decoder that decodes \{hello: world\} and \{hello:
32\}. Learned some DCG and rewrote the parser with it.
\item March 30. Researched how SWI interacts with C. Wrote a bytes8\_to\_double
function in C and managed to integrate it with the rest of the system.
Separated bson into separate files, encoder and decoder. Should think
about researching PlUnit by now. Thought about and tried PlUnit. Works
fine. But how to organize tests? Implemented all bit-hacking in C
for the time being. Works. Simple. Fun. Started looking at PlDoc.
Generating actual docs seems like a hassle, probably won't do it.
But the important thing is unit tests, and some comments inside the
source. Started fleshing out the report slightly, with some ideas
and stubs.
\item March 31. Restructured project folder. Better including of modules
now, I think. Starting to write nice unit tests. Continued with the
decoder a little, the complex BSON example works. Wrote nice TextMate
snippets for test boilerplating. Rewrote Makefile, yet again. Added
make test, for instance.
\item April 1. When trying to fix proper decoding of strings I ran into
issues with Unicode. Finally found SWI-Prolog's memory files, which
can be written to and then read back using a different encoding. Five
line fix. Implemented a variant that might be more efficient. Might.
(And like five times longer, but clean though.) Cleaned up the tests.
Implementing the rest of the decoder should be straight-forward now.
Next week. It. Will. Be. Awesome.
\item April 2. Looked into the Prolog module system again and started adding
prefixes ({}``memory\_file:'', {}``builtin:'') to SWI predicate
calls. {}``builtin'' isn't an actual module, but since it works
anyway (it enters the module builtin and calls the predicate from
there) I decided to use it. Documented predicates in bson\_bits. Fixed
a bug with bytes\_to\_integer/5 because it wasn't using a 32-bit integer.
Now it uses int32\_t. Started using maplist/2. Found a library(apply\_macros)
that claims to speed up maplist etc. simply by loading it.
\item April 3. All use\_module directives now use an empty list in order
to ensure that the module is loaded, but nothing is imported. This
forces me to add the appropriate module prefixes to calls. Benchmarked
the unicode converter and the old shorter code outperformed the new
longer code with more than a factor of three. I guess this is due
to more stuff being done by the C library and not Prolog. Anyway,
the old code is faster and much shorter -- win-win. Also replaced
a crappy to-codes-and-then-to-atom conversion with a straight-to-utf8-atom
library routine, which made it even faster. A keeper.
\item April 4. Refactored decoder slightly. Added relation term\_bson/2
to bson, which calls necessary subpredicate. Moved tests into separate
files: they were getting too long, and separating them makes it easier
to write them simultaneously. As long as tests and implementation
are close to each other, I am okay. (Impl in .pl and tests in .plt
next to each other.) The decoder should now be feature complete. It
does minimal error checking, but it should be able to parse all element
types. It might still need some cleaning though. Added version predicates
to bson. Made comments more in line with PlDoc.
\item April 5. Major decoder cleanup. Took a slow day.
\item April 6. Started on the encoder. Unicode issues again. Solved it after
an hour or so, using even more library predicates. Factored out Unicode
handling to separate module, used by both encoder and decoder. Worked
a lot on the encoder, including some C hacking again.
\item April 7. Lots more encoding of elements and encoder refactoring. More
to come!
\item April 8. Clean up a lot of bit-hacking in Prolog. Made entry-predicates
into true relations and made the actual conversions into more general
list predicates. Changed {}``builtin:'' prefix to {}``inbuilt:'',
which looks nicer and is more of a single word. Real cleanup in all
the bit-hacking. It sure took must of the day, but things should be
pretty solid and clean now, and the strategy is flexible but straight-forward:
all integer conversions back-and-forth between signed big/little 32/64-bit
and bytes is done in C. If those routines cannot be used, it fails
if a negative integer is being encoded, otherwise it is treated as
an unbounded unsigned, making for instance ObjectID simple. Everything
goes through either the relation \noun{float\_bytes(Float, Bytes)}
which is just mappings to C, or the more intricate relation \noun{integer\_bytes(Integer,
NumBytes, Endian, Bytes)} which can be called like: (+,+,+,?) or (?,+,+,+).
E.g. (1, 4, little, Bytes) gives Bytes = {[}1,0,0,0{]}, or (Integer,
4, big, {[}0,0,0,1{]}) gives Integer = 1. The first example can be
read like {}``the integer 1 in 4 little(-endian) bytes.''
\item April 9. Factored out the unbounded unsigned handling into a separate
predicate \noun{unsigned\_bytes(Unsigned, NumBytes, Endian, Bytes)}.
Now things are starting to look really good. float\_bytes/2 handles
doubles, integer\_bytes/4 handles all signed 32/64-bit integers and
unsigned\_bytes/4 handles all unbounded unsigneds. Simple. Also documented
the predicates.
\item April 10. Made all Unicode handling atom-only. The code became much
less confusing, and using code lists for text would be bad anyway
because it can be ambiguous (is {[}97,98,99{]} a list of numbers or
'abc'?). Yet another implementation of utf\_to\_bytes/2. Uses fewer
weird predicates (no open\_chars\_stream any more) and seems a tiny
bit faster. The code is a bit longer though (but arguably more straight-forward).
\item April 11. Encoder is now feature complete. Tomorrow I will start to
think about the report and maybe, just slightly, investigate sockets.
\item April 12. Started hacking a little on the report structure. Once again
renamed {}``inbuilt:'' to {}``core:''. Nicer and shorter. Started
looking at sockets. Created mongo module stubs.
\end{itemize}

\section{Introduction}

From spec:

{[}MongoDB is a young document-oriented database system that has started
to gain much attention recently. Document-orientation involves removing
rigid database schemas and advanced transactions, in favor of flexibility.
Document-orientation also promotes a certain degree of denormalization
which allows embedding documents into each other, leading to potentially
much better performance by avoiding the need for expensive join operations.

Prolog, being an untyped language, agrees with the document-oriented
approach of relaxing manifests in order to create more dynamic and
flexible systems. Embedding terms in other terms is natural in Prolog,
and embedding documents in other documents is natural in MongoDB.

Many drivers exist, both official and unofficial, that enable the
use of Mongo\-DB from various programming languages. At the time
of writing, no such driver for Prolog seems to exist.{]}


\section{Scope}

From spec:

{[}Creating a driver that is usable with, or at least easily portable
to, other Prolog environments is desirable, but development and testing
will focus on SWI-Prolog.

More complex features of Mongo\-DB, such as advanced connection management
and GridFS, will not be implemented.{]}


\section{Method (necessary?)}

Research other drivers, docs.

Test-driven development.


\section{Requirements}

Various requirements on the implementation. See for example

http://www.mongodb.org/display/DOCS/Writing+Drivers+and+Tools


\section{Design}
\begin{itemize}
\item BSON encoder/decoder

\begin{itemize}
\item Some parts written in C. Why? (Basically didn't know how to easily
handle bytes-to-float in Prolog. And perhaps some efficiency.)
\item Discuss data structures, term {[}key:value{]} maps to bytelist {[}4,1,7,9,3,...{]}
etc.
\item Design choices: text as atoms (why not list of codes, {[}97,98,99{]}?)
Inspired slightly by JSON parser: http://www.swi-prolog.org/pldoc/doc\_for?object=section\%283,\%275.1\%27,swi\%28\%27/doc/packages/http.html\%27\%29\%29
\end{itemize}
\item Network communication

\begin{itemize}
\item How does the communication work? How simplistic is the communication
administration? Sockets, connections, etc.
\end{itemize}
\item MongoDB API

\begin{itemize}
\item Thoroughly discuss design of wrapper API, how lists and structures
are represented etc. What algorithms are used etc.
\end{itemize}
\end{itemize}

\section{Implementation}

How do I solve it?


\section{Evaluation}

Did it work? Is it usable? What should have been done differently?


\section{Related Work}

Compare to existing drivers? Erlang? And something completely different?


\section{Conclusion/Future Work}

Not sure how much this section relates to Evaluation above.

Portability (Tested on Mac, SWI, GCC/Clang, etc.)? Efficiency? How
to improve in the future? Critical parts (BSON) in pure C? Even write
a C extension with more/most functionality? Don't know how portable
that would be though, but SWI (and probably SICStus also?) has a mature
interface to C. What is missing for it to be a {}``real'' driver?


\section{References}

References for Prolog DCG? References for various stuff used in SWI?
At all relevant? How much web references? Most MongoDB driver stuff
is web.

Chodorow, K. \& Dirolf, M. (2010) \emph{MongoDB: The Definitive Guide.}
Sebastopol, United States of America: O'Reilly Media, Inc.
\end{document}
